%%%%%%%%%%%%%%%%%%%%% chapter.tex %%%%%%%%%%%%%%%%%%%%%%%%%%%%%%%%%
%
% sample chapter
%
% Use this file as a template for your own input.
%
%%%%%%%%%%%%%%%%%%%%%%%% Springer-Verlag %%%%%%%%%%%%%%%%%%%%%%%%%%

\chapter{Resumo}
\label{abstract} % Always give a unique label
% use \chaptermark{}
% to alter or adjust the chapter heading in the running head

Este artigo foi elaborado no contexto do curso de Programação em Lógica, e incide sobre a programação em lógica com restrições. 
A programação em lógica com restrições permite restringir problemas por domínio de soluções e por condições que devem ser cumpridas. 

Com o objectivo de avaliar a viabilidade da programação com restrições para resolver problemas de lógica de complexidade relevante, foram desenvolvidos algoritmos de resolução e de geração de novas soluções para dimensões variáveis do jogo Turn 12.
Para o efeito fez-se recurso à biblioteca de restrições para o cojunto dos domínios finitos clp(FD) do SICStus Prolog.

O método utilizado compreende a rotação dos dígitos de cada face, sendo o domínio o conjunto de rotações possíveis. A combinação das faces é feita iterativamente e não simultâneamente, por forma a optimizar os recursos.

Obtiveram-se resultados em tempo útil para solucionar o problema original e para um número de digitos não superior a 60, o que permitiu concluir que a metodologia utilizada é adequada para a resolução de problemas do género.
